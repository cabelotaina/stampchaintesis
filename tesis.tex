% online - https://pt.sharelatex.com/project/54c8c145ce57ee9e636bf385

% base do documento https://projetos.inf.ufsc.br/arquivos/Padrao_de_Proposta.pdf
% apoio latex https://github.com/luanadfreitas/latex
\documentclass[12pt,openright,twoside,a4paper,english, brazil]{abntex2} % com draft as imagens desaparecem
\usepackage[utf8]{inputenc}
\usepackage[brazil]{babel}
\usepackage{indentfirst}
\usepackage{hyperref}
\usepackage{graphicx}
\usepackage{pdfpages}
\usepackage{tabularx}
\usepackage[alf]{abntex2cite} % Citações padrão ABNT
%\usepackage{lscape} % pacote para colocar página em paisagem
\usepackage{color, colortbl} % colorir tabela
\usepackage{multirow} % várias colunas
\usepackage{verbatim} % usado pelo begin/end{comment}
\usepackage{tikz-uml}
\usepackage{usecases}

%Siglas precisa desses pacotes
\usepackage{datagidx}
\newgidx{glossary}{Lista de Abreviaturas e Siglas}
\DTLgidxSetDefaultDB{glossary}

% Siglas
\newterm[description={Trabalho de Conclusão de Curso}]{TCC}
%\newterm[description={GNU's not Unix}]{GNU}
%\newterm[description={GNU public license (Licença pública GNU)}]{GPL}
%\newterm[description={Model, View and Control (Modelo, Visão e Controle)}]{MVC}
%\newterm[description={C reate, R ead, U pdate e D elete}]{CRUD}
\newterm[description={JavaScript}]{JS}
\newterm[description={Universidade Federal de Santa Catarina}]{UFSC}
\newterm[description={global positioning system}]{GPS}


\makeindex



% folha de rosto
\title{Registro Eletrônico de Ponto Seguro baseado em Blockchain}
\author{Maurilio Atila Carvalho de Santana}
\orientador{Ricardo Felipe Custódio}
\instituicao{Universidade Federal de Santa Catarina}
\local{Florianópolis}
\data{Fevereiro de 2018}

\begin{document}
\maketitle

%\begin{folhadeaprovacao}
%Trabalho sob o título \textit{Registro Eletrônico de Ponto Seguro baseado em Blockchain}, defendido por Maurilio Atila Carvalho de Santana e aprovado para obtenção do grau de Bacharel em Ciências da Computação em 28 de Junho de 2018, em Florianópolis,
%Estado de Santa Catarina, pela banca examinadora constituída pelos
%membros:

%\setlength{\ABNTsignthickness}{0.4pt}
%\setlength{\ABNTsignwidth}{10cm}
%\setlength{\ABNTsignskip}{3.5cm}

%\assinatura{Ricardo Felipe Custódio \\ Orientador}
%\assinatura{Daniel Neis Araujo}
%\assinatura{...}
%\end{folhadeaprovacao}

%\capa

\folhaderosto
\includepdf[pages={1}]{figuras/folha_de_aprovacao_proposta_tcc.pdf}	

\begin{resumo}

Este trabalho de conclusão de curso propõe uma aplicação mobile para o Registro Eletrônico de Ponto Seguro baseado em Blockchain e em tecnologias como GPS e o QR-CODE. O objetivo da ferramenta é permitir uma maior segurança tanto por parte do empregador quanto por parte do trabalhador, dado que mesmo com a portaria 1510 do MTE, que especifica o funcionamento dos relogios de ponto, ainda existem muitas fraudes e desconfiança quando a segurança, tanto por parte dos empregados, quanto por parte dos empregadores. Esperamos que este trabalho ofereça uma contribuição pontual para a discução do uso de blockchain pelo poder publico.

\textbf{Palavras-chave: } Blockchain, Registro Eletrônico de Ponto, GPS, QR-CODE

\end{resumo}

% \begin{abstract}
%
% Este trabalho de conclusão de curso propõe uma aplicação mobile para o Registro Eletrônico de Ponto Seguro baseado em Blockchain e em tecnologias como GPS e o QR-CODE. O objetivo da ferramenta é permitir uma maior segurança tanto por parte do empregador quanto por parte do trabalhador, dado que mesmo com a portaria 1510 do MTE, que especifica o funcionamento dos relogios de ponto, ainda existem muitas fraudes e desconfiança quando a segurança, tanto por parte dos empregados, quanto por parte dos empregadores. Esperamos que este trabalho ofereça uma contribuição pontual para a discução do uso de blockchain pelo poder publico.
%
% \textbf{Keywords: } Blockchain, Registro Eletrônico de Ponto, GPS, QR-CODE
%
% \end{abstract}

% Lista de figuras
\pdfbookmark[0]{\listfigurename}{lof}
\listoffigures*
\cleardoublepage %pula uma página
\pdfbookmark[0]{\listtablename}{lot}
\listoftables* % Lista de tabelas
\cleardoublepage
\printterms[columns=1,style=align] % lista de abreviaturas e siglas
\cleardoublepage % pula uma página
%\listadesimbolos
\tableofcontents








\chapter{Introdução}



Neste capitulo vamos apresentar um descrição dos objetivos deste trabalho explicitando suas partes e tecnologias envolvidas, o que um registro de ponto, desde de quanto ele vem sendo utilizado no Brasil, vamos apresentar as portarias que regulamentam seu uso.

\section{O Trabalho}

Este trabalho tem o intuito de permitir o registro de ponto pelo empregador e trabalhador por meio de um smartphone, a ideia é ter um app que obtenha coordenadas ou um qr-code, a hora certa do aparelho e de uma entidade que ofereça o serviço de carimbos do tempo e enviar para o backend do sistema.

O backend deve utilizar parte das estruturas de dados previstas na portaria 1.510 do MTE, que regulamenta o registro de ponto, com o intuito de estar de acordo com outros relogios de ponto reais. Essas estruturas de dados serão visiveis porem somente de leitura por parte das empresas, para isso vamos contar com um contrato inteligente que será usado como folha de ponto, cada vez que um funcionário entrar ou sair da empresa as estruturas de dados serão modificadas para refletir a entrada ou saida desse trabalhador.

Além disso outra ferramenta será um webapp com um dashboard mostrando como esta a frequencia de cada funcionario. Os funcionários por sua vez podem ver sua frequencia e todo seu histórico de trabalho por meio do aplicativo no seu smartphone.

O foco principal dessa ferramenta é apoiar o trabalhador e o empregador em situações de falha de um equipamnento de registro de ponto, bem como em situações de desentendimento entre ambos.

\section{Registro de Ponto}

O registro de ponto foi criado em 20 de Novembro de 1888 por Willard Le Grand Bundy um joallheiro de Auburn, Nova Iorque ele criou em 1889 a International Time Recording. Abaixo a imagem de sua patente:

https://www.google.com/patents/US452894

Seu irmão entrou no negocio e começaram a fabricar relogios em grande escala, pela Bundy Manufacturing Company, notavelmente após alguns anos em 1911 eles mantinham a empresa Computing-Tabulating-Recording Company (CTR) o precursor da IBM.

Um relogio de ponto básico apenas carimba a data e o horario em cartões dos funcionarios. Mas haviam relogios que foram criados por essa empresa de Bundy que continham por exemplo uma chave para cada funcionario, evitando fraudes. Hoje em dia existem relogios de ponto que utilizam biometria, e mesmo assim são burlados por meio de dedos de silicone por exemplo.

\section{Regulamentação do Mninistério Trabalho e Emprego}

O registro de ponto no Brasil é regulamentado portaria do Ministério do Trabalho e emprego MTE Nº 1.510, de 21 de Agosto de 2009 e entrou vigência plena em 2011, o registro de ponto é obrigatorio e não deve em hiposete alguma ser alterado pelo empregador, existe um conjunto de empresas homologadas pelo MTE que tem permissão para oferecer estes equipamentos que segue um conjunto de normas especificas descritas na portaria acima citada.

Além disso alguns tecnicos podem oferecer serviços de manutenção para o equipamento. Os equipamentos contam com relogio interno capaz de funcionar mesmo em situações de falta de energia, garante um horario com alta precisão bem como é protegido por meio de lacres e vários tipos de mecanismos contra adulteração, é realizada impressão interna em papel e existe uma memoria interna permanente, as estruturas de dados para armazenamento da informação são descritos nos anexos do documento.

O equipamento possui uma porta USB, de auditoria, para que os fiscais do MTE possam realizar processos de vistoria nos equipamentos a qualquer momento. Em casos de adulteração são tomadas as medidas civeis e criminais definidas na legislação. 

Mas mesmo com todos esses mecanismos de proteção a ainda existe insegurança tanto por parte dos empregadores como dos empregados.

% Fale rapidamente sobre os requisitos de segurança impostos pela Portaria 1510

%\section{Objetivos específicos}

%O objetivo geral é criar ....

%\begin{itemize}
% \item Desenvolver ...
%\end{itemize}

%\section{Justificativa}

%\section{Metodologia}

%Análise de papers em bases de artigos cientificos. Leitura dos artigos similares a temática escolhida, e artigos relacionados aos conhecimentos basais necessários, para escrita deste projeto.

%\section{Estrutura do Trabalho}







\chapter{Registro Eletronico de Ponto}

Use a Portaria 1510 e a 373 para isso. 

Também veja os mais diversos sites na Internet sobre produto, sistemas etc.... e suas consideracoes

Veja exemplos de apps para registro de ponto

\section{Requisitos funcionais}

\section{Requisitos não funcionais}

\section{Sistemas de registro de ponto ( SREP )}

%\section{Trabalhos Relacionados}








\chapter{Tecnologias Relacionadas}

\section{Blockchain}

\section{GPS}

\section{QR-Code}






\chapter{App Universal par Registro de Ponto}

Neste capitulo apresentamos alguns wireframes que permitem entender o fluxo de navegação que propomos para este app.

\section{Tela Inicial}

\section{Leitor de QR-CODE}

\section{Configurações do Trabalhador}

\section{Compartilhamento de informações}





\chapter{Prova de Conceito e Análise}

Vamos colocar o aplicativo no mercado e avaliar a sua utilizacao





\chapter{Consideracoes Finais}

Escrever sobre o que foi feito e que no futuro irá integrar-se o registro de ponto seguro a um problema maior, que é a carteira de trabalho eletronica.


\bibliography{bibliography}

\apendices

\end{document}