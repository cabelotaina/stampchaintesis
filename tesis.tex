% online - https://pt.sharelatex.com/project/54c8c145ce57ee9e636bf385

% base do documento https://projetos.inf.ufsc.br/arquivos/Padrao_de_Proposta.pdf
% apoio latex https://github.com/luanadfreitas/latex
\documentclass[12pt,openright,twoside,a4paper,english, brazil]{abntex2} % com draft as imagens desaparecem
\usepackage[utf8]{inputenc}
\usepackage[brazil]{babel}
\usepackage{indentfirst}
\usepackage{hyperref}
\usepackage{graphicx}
\usepackage{pdfpages}
\usepackage{tabularx}
\usepackage[alf]{abntex2cite} % Citações padrão ABNT
%\usepackage{lscape} % pacote para colocar página em paisagem
\usepackage{color, colortbl} % colorir tabela
\usepackage{multirow} % várias colunas
\usepackage{verbatim} % usado pelo begin/end{comment}
\usepackage{tikz-uml}
\usepackage{usecases}

%Siglas precisa desses pacotes
\usepackage{datagidx}
\newgidx{glossary}{Lista de Abreviaturas e Siglas}
\DTLgidxSetDefaultDB{glossary}

% Siglas
\newterm[description={Trabalho de Conclusão de Curso}]{TCC}
%\newterm[description={GNU's not Unix}]{GNU}
%\newterm[description={GNU public license (Licença pública GNU)}]{GPL}
%\newterm[description={Model, View and Control (Modelo, Visão e Controle)}]{MVC}
%\newterm[description={C reate, R ead, U pdate e D elete}]{CRUD}
\newterm[description={JavaScript}]{JS}
\newterm[description={Universidade Federal de Santa Catarina}]{UFSC}
\newterm[description={global positioning system}]{GPS}


\makeindex



% folha de rosto
\title{Registro Eletrônico de Ponto Seguro baseado em Blockchain}
\author{Maurilio Atila Carvalho de Santana}
\orientador{Ricardo Felipe Custódio}
\instituicao{Universidade Federal de Santa Catarina}
\local{Florianópolis}
\data{Fevereiro de 2018}

\begin{document}
\maketitle

%\begin{folhadeaprovacao}
%Trabalho sob o título \textit{Uma aplicação para o mapeamento de ocorrências e sugestões baseada no FixMyStreet e OpenStreetMap}, defendido por Maurilio Atila Carvalho de Santana e aprovado para obtenção do grau de Bacharel em Ciências da Computação em 28 de maio de 20015, em Florianópolis,
%Estado de Santa Catarina, pela banca examinadora constituída pelos
%membros:

%\setlength{\ABNTsignthickness}{0.4pt}
%\setlength{\ABNTsignwidth}{10cm}
%\setlength{\ABNTsignskip}{3.5cm}

%\assinatura{Prof. Antônio Carlos Mariani \\ Orientador}
%\assinatura{Prof. Daniel Neis Araujo}
%\assinatura{Prof. Everton da Silva }
%\end{folhadeaprovacao}

%\capa

\folhaderosto
\includepdf[pages={1}]{figuras/folha_de_aprovacao_proposta_tcc.pdf}	

\begin{resumo}

Este trabalho de conclusão de curso propõe uma aplicação mobile para o Registro Eletrônico de Ponto Seguro baseado em Blockchain e em tecnologias como GPS e o QR-CODE. O objetivo da ferramenta é permitir uma maior segurança tanto por parte do empregador quanto por parte do trabalhador, dado que mesmo com a portaria 1510 do MTE, que especifica o funcionamento dos relogios de ponto, ainda existem muitas fraudes e desconfiança quando a segurança, tanto por parte dos empregados, quanto por parte dos empregadores. Esperamos que este trabalho ofereça uma contribuição pontual para a discução do uso de blockchain pelo poder publico.

\textbf{Palavras-chave: } Blockchain, Registro Eletrônico de Ponto, GPS, QR-CODE

\end{resumo}

% \begin{abstract}
%
% Este trabalho de conclusão de curso propõe uma aplicação mobile para o Registro Eletrônico de Ponto Seguro baseado em Blockchain e em tecnologias como GPS e o QR-CODE. O objetivo da ferramenta é permitir uma maior segurança tanto por parte do empregador quanto por parte do trabalhador, dado que mesmo com a portaria 1510 do MTE, que especifica o funcionamento dos relogios de ponto, ainda existem muitas fraudes e desconfiança quando a segurança, tanto por parte dos empregados, quanto por parte dos empregadores. Esperamos que este trabalho ofereça uma contribuição pontual para a discução do uso de blockchain pelo poder publico.
%
% \textbf{Keywords: } Blockchain, Registro Eletrônico de Ponto, GPS, QR-CODE
%
% \end{abstract}

% Lista de figuras
\pdfbookmark[0]{\listfigurename}{lof}
\listoffigures*
\cleardoublepage %pula uma página
\pdfbookmark[0]{\listtablename}{lot}
\listoftables* % Lista de tabelas
\cleardoublepage
\printterms[columns=1,style=align] % lista de abreviaturas e siglas
\cleardoublepage % pula uma página
%\listadesimbolos
\tableofcontents








\chapter{Introdução}

\section{Registro de Ponto}

O registro de ponto no Brasil é regulamentado pela lei ... e existe desde ...

O registro de ponto surgiu em ... sendo utilizado para ...

% Fale rapidamente sobre os requisitos de segurança impostos pela Portaria 1510

%\section{Objetivos específicos}

%O objetivo geral é criar ....

%\begin{itemize}
% \item Desenvolver ...
%\end{itemize}

%\section{Justificativa}

%\section{Metodologia}

%Análise de papers em bases de artigos cientificos. Leitura dos artigos similares a temática escolhida, e artigos relacionados aos conhecimentos basais necessários, para escrita deste projeto.

%\section{Estrutura do Trabalho}








\chapter{Registro Eletronico de Ponto}

Use a Portaria 1510 e a 373 para isso. 

Também veja os mais diversos sites na Internet sobre produto, sistemas etc.... e suas consideracoes

Veja exemplos de apps para registro de ponto

\section{Requisitos funcionais}

\section{Requisitos não funcionais}

\section{Sistemas de registro de ponto ( SREP )}

%\section{Trabalhos Relacionados}








\chapter{Tecnologias Relacionadas}

\section{Blockchain}

\section{GPS}

\section{QR-Code}






\chapter{App Universal par Registro de Ponto}

Neste capitulo apresentamos alguns wireframes que permitem entender o fluxo de navegação que propomos para este app.

\section{Tela Inicial}

\section{Leitor de QR-CODE}

\section{Configurações do Trabalhador}

\section{Compartilhamento de informações}





\chapter{Prova de Conceito e Análise}

Vamos colocar o aplicativo no mercado e avaliar a sua utilizacao





\chapter{Consideracoes Finais}

Escrever sobre o que foi feito e que no futuro irá integrar-se o registro de ponto seguro a um problema maior, que é a carteira de trabalho eletronica.


\bibliography{bibliography}

\apendices

\end{document}